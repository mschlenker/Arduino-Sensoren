\section{Arduino vs. RaspberryPi} 

\begin{frame}
\frametitle{Worin unterscheiden sich die beiden Bastlerplatinen?} 
Auf den ersten Blick bietet der Raspberry Pi mehr als 1000 mal soviel Rechenleistung wie ein Arduino Uno f�r zehn Euro Aufpreis. Da f�llt die Entscheidung leicht, oder?

Sehen wir uns die Unterschiede doch einmal n�her an:
\end{frame}

\subsection{Technische Daten}
\begin{frame}
\frametitle{Speicher und Rechenleistung} 
\begin{tabular}{c c c c}
\textbf{Plattform} & \textbf{Arduino} & \textbf{Raspberry Pi} & \textbf{Faktor} \\ 
\pause 
RAM/Variablenspeicher & 2048 & 536870912 & 262144\\
\pause 
Festspeicher & 32768 & 8589934592 & 262144 \\
\pause 
,,Rechenleistung'' (MIPS) &  16 & ~ 1000 & 64 \\
\end{tabular} 
\end{frame}

\begin{frame}
\frametitle{IO und Buses}
Arduino heisst in diesem Kontext Atmega328P, Raspberry Pi meint das Model B...
\newline

\begin{tabular}{c c c c}
\textbf{Plattform} & \textbf{Arduino} & \textbf{Raspberry Pi} & \textbf{Faktor} \\ 
\pause
GPIO digital & 17 & 8 & unfair! �pfel vs Birnen! \\
\pause
analog & 6 & 0 & 0 \\
\pause
SPI & 1 & 2 & 2 \\
\pause
UART (seriell) & 1 & 1 & 1 \\
\pause
I2C & 1 & 1 & 1 \\
\end{tabular} 
\end{frame}

\begin{frame}
\frametitle{Interrupts} 
Arduino heisst in diesem Kontext Atmega328P, Raspberry Pi meint das Model B...
\newline

\begin{tabular}{c c c c}
\textbf{Plattform} & \textbf{Arduino} & \textbf{Raspberry Pi} & \textbf{Faktor} \\ 
\pause
Steigend/Fallend & 2 & 8? Publikum? & ? \\
\pause
Wakeup-Timer & 2 & beliebig viele & ? \\
\end{tabular} 


\end{frame}

\begin{frame}
\frametitle{Leistungsaufnahme} 
\end{frame}

\subsection{Einsatzbereiche}

\begin{frame}
\frametitle{Arduino} 
\end{frame}

\begin{frame}
\frametitle{Raspberry Pi} 
\end{frame}