%  
% � Mattias Schlenker 
% Dieser Text ist nicht frei, bitte wegen CC-Lizenzierung anfragen!
% 


\documentclass{beamer}
%\usepackage[german]{babel}
% \usepackage[iso-8859-1]{inputenc}
\usepackage[latin9]{inputenc}
\setbeamertemplate{navigation symbols}{}
\usetheme{Warsaw}

\beamersetuncovermixins{\opaqueness<1>{25}}{\opaqueness<2->{15}}
\begin{document}
\title{Mir gen�gt ein Arduino! Microcontroller statt Prozessor}  
\author{Mattias Schlenker}
\date{28. Juni 2014} 

% Titelseite 

\begin{frame}
\titlepage
\end{frame}

% TOC-Seite

\begin{frame}
\frametitle{Inhalt}
\tableofcontents
\end{frame} 


\section{Arduino vs. RaspberryPi} 

\begin{frame}
\frametitle{Worin unterscheiden sich die beiden Bastlerplatinen?} 
Auf den ersten Blick bietet der Raspberry Pi mehr als 1000 mal soviel Rechenleistung wie ein Arduino Uno f�r zehn Euro Aufpreis. Da f�llt die Entscheidung leicht, oder?

Sehen wir uns die Unterschiede doch einmal n�her an:
\end{frame}

\subsection{Technische Daten}
\begin{frame}
\frametitle{Speicher und Rechenleistung} 
\end{frame}

\begin{frame}
\frametitle{IO und Buses} 
\end{frame}

\begin{frame}
\frametitle{Interrupts} 
\end{frame}

\begin{frame}
\frametitle{Leistungsaufnahme} 
\end{frame}

\subsection{Einsatzbereiche}

\begin{frame}
\frametitle{Arduino} 
\end{frame}

\begin{frame}
\frametitle{Raspberry Pi} 
\end{frame}

\section{Die Arduino-Zukunft}

\subsection{,,Starke'' Arduinos mit Linux}
\subsection{Neue Microcontroller mit ARM-Kern}

\section{Gemeinsam stark!}

\subsection{Raspberry Pi mit Arduino huckepack}
\subsection{Raspberry Pi als Zentrale}


%\subsection{Subsection no.1.1  }

\begin{frame} 
Without title somethink is missing. 
\end{frame}


%\section{Section no. 2} 
%\subsection{Lists I}
\begin{frame}\frametitle{unnumbered lists}
\begin{itemize}
\item Introduction to  \LaTeX  
\item Course 2 
\item Termpapers and presentations with \LaTeX 
\item Beamer class
\end{itemize} 
\end{frame}

\begin{frame}\frametitle{lists with pause}
\begin{itemize}
\item Introduction to  \LaTeX \pause 
\item Course 2 \pause 
\item Termpapers and presentations with \LaTeX \pause 
\item Beamer class
\end{itemize} 
\end{frame}

\subsection{Lists II}
\begin{frame}\frametitle{numbered lists}
\begin{enumerate}
\item Introduction to  \LaTeX  
\item Course 2 
\item Termpapers and presentations with \LaTeX 
\item Beamer class
\end{enumerate}
\end{frame}

\begin{frame}\frametitle{numbered lists with pause}
\begin{enumerate}
\item Introduction to  \LaTeX \pause 
\item Course 2 \pause 
\item Termpapers and presentations with \LaTeX \pause 
\item Beamer class
\end{enumerate}
\end{frame}

%\section{Section no.3} 
%\subsection{Tables}
\begin{frame}\frametitle{Tables}
\begin{tabular}{|c|c|c|}
\hline
\textbf{Date} & \textbf{Instructor} & \textbf{Title} \\
\hline
WS 04/05 & Sascha Frank & First steps with  \LaTeX  \\
\hline
SS 05 & Sascha Frank & \LaTeX \ Course serial \\
\hline
\end{tabular}
\end{frame}


\begin{frame}\frametitle{Tables with pause}
\begin{tabular}{c c c}
A & B & C \\ 
\pause 
1 & 2 & 3 \\  
\pause 
A & B & C \\ 
\end{tabular} 
\end{frame}


%\section{Section no. 4}
%\subsection{blocs}
\begin{frame}\frametitle{blocs}

\begin{block}{title of the bloc}
bloc text
\end{block}

\begin{exampleblock}{title of the bloc}
bloc text
\end{exampleblock}


\begin{alertblock}{title of the bloc}
bloc text
\end{alertblock}
\end{frame}

%\section{Section no. 5}
%\subsection{split screen}

\begin{frame}\frametitle{splitting screen}
\begin{columns}
\begin{column}{5cm}
\begin{itemize}
\item Beamer 
\item Beamer Class 
\item Beamer Class Latex 
\end{itemize}
\end{column}
\begin{column}{5cm}
\begin{tabular}{|c|c|}
\hline
\textbf{Instructor} & \textbf{Title} \\
\hline
Sascha Frank &  \LaTeX \ Course 1 \\
\hline
Sascha Frank &  Course serial  \\
\hline
\end{tabular}
\end{column}
\end{columns}
\end{frame}

\subsection{Pictures} 
\begin{frame}\frametitle{pictures in latex beamer class}
\begin{figure}
% \includegraphics[scale=0.5]{PIC1} 
\caption{show an example picture}
\end{figure}
\end{frame}

\subsection{joining picture and lists} 

\begin{frame}
\frametitle{pictures and lists in beamer class}
\begin{columns}
\begin{column}{5cm}
\begin{itemize}
\item<1-> subject 1
\item<3-> subject 2
\item<5-> subject 3
\end{itemize}
\vspace{3cm} 
\end{column}
\begin{column}{5cm}
\begin{overprint}
%\includegraphics<2>{PIC1}
%\includegraphics<4>{PIC2}
%\includegraphics<6>{PIC3}
\end{overprint}
\end{column}
\end{columns}
\end{frame}


\subsection{pictures which need more space} 
\begin{frame}[plain]
\frametitle{plain, or a way to get more space}
\begin{figure}
% \includegraphics[scale=0.5]{PIC1} 
\caption{show an example picture}
\end{figure}
\end{frame}



\end{document}
